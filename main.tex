
\documentclass[12pt]{article}
\usepackage[margin=1in]{geometry}
\usepackage[english]{babel}
\usepackage[utf8x]{inputenc}
\usepackage{amsmath}
\usepackage{amsfonts}
\usepackage{amssymb}
\usepackage{parskip}
\usepackage{rotating}
\usepackage{natbib}
\usepackage{subfigure}
\usepackage{tikz}
\usepackage{graphicx}
%% PATH TO FIGURES and TABLES
\graphicspath{ {./Exhibits/} }

\usepackage{watermark}
\thiswatermark{\centering \put(-70,-749){\includegraphics[scale=1]{cover.png}} }
\usepackage{titlesec}
\usepackage{xcolor}
\titleformat*{\section}{\filcenter\Large\bfseries\sffamily\color{red}}
\titleformat*{\subsection}{\large\bfseries\sffamily\color{red}}
\titleformat*{\subsubsection}{\sffamily\color{red}}
\usepackage{fancyhdr}
\pagestyle{fancy}
\usepackage{pdfpages}
\usepackage{apacite}

%% ENTER SHORT TITLE for RUNNING HEAD
\lhead{\textsf{ }} % controls the left corner of the header
%%
\cfoot{} % controls the center of the footer
\rhead{\textsf{Page~\thepage}} % controls the right corner of the footer
 

\begin{document}


\thispagestyle{empty}
\cleardoublepage{}

\hspace{-2cm}\begin{minipage}[b]{0.3\linewidth}
\begin{flushleft}
 
\vspace{6cm}

%%% ENTER AUTHOR NAMES HERE   
\textsf{ } \\
\textsf{ } \\
\end{flushleft}
\end{minipage}

\hspace{5cm}\begin{minipage}[t]{0.8\textwidth}
\begin{center}

%%% ENTER TITLE HERE
\Large{\textbf{\textsf{WI22-QTP: Economic Security of People with Disabilities during the Pandemic }}}\\
\end{center}
\end{minipage}

\hspace{5cm}\begin{minipage}[t]{0.7\textwidth}


\begin{flushleft}
\vspace{6cm}

%% STANDARD DISCLAIMER
\footnotesize{\textsf{The research reported herein was performed pursuant to a grant from the U.S. Social Security Administration (SSA) funded as part of the Retirement and Disability Consortium. The opinions and conclusions expressed are solely those of the author(s) and do not represent the opinions or policy of SSA or any agency of the Federal Government. Neither the United States Government nor any agency thereof, nor any of their employees, makes any warranty, express or implied, or assumes any legal liability or responsibility for the accuracy, completeness, or usefulness of the contents of this report. Reference herein to any specific commercial product, process or service by trade name, trademark, manufacturer, or otherwise does not necessarily constitute or imply endorsement, recommendation or favoring by the United States Government or any agency thereof.}}
\end{flushleft}

\end{minipage}
 


\cleardoublepage{}

% SPACING 
\linespread{1.25} 
\section*{Abstract}
		\noindent   
		
				\noindent\textbf{JEL Classification Codes:} I31, I32, R21. \\
		\noindent\textbf{Keywords:}    Social Security\\

\newpage 
 
\section{Introduction}


\includegraphics[scale=0.65]{}


\subsection{Background}

 
 
 
There are multiple paths through which the COVID-19 pandemic may affect the economic security of working age adults with disabilities. Disruptions to the economy and risk of disease spread led some who were working prior to the pandemic to exit the labor market \citep{cheng2020back, goda2021impact moen2020disparate, quinby2021older }, thereby reducing income through wage earnings. This reduction in labor force participation has not been offset by a subsequent increase in Social Security  claiming [CITE?].   


Reduced consumption \citep{baker2020does,horvath2021covid} 

households often unable to pay their bills \citep{clark2021financial,schneider2020household}. 

  
We begin our analysis by  

We further explore mechanisms underlying changes 

 

Taken together, the results of this study provide important detail about the depth and breadth of the inequity of financial hardship experienced in the pandemic. Our findings indicate heterogeneous effects for vulnerable segments of the population.    


\cite{adams2020inequality,andersen2020consumer,andersen2020pandemic,baker2020does,bhutta2020covid,braxton2020can,brewer2020initial,casado2020aggregate,cheng2020back,cherry2021government,chetty2020did,clark2021financial,coibion2020labor,cfpb2020,cowan2020short,cutts2020musings,emerson2021impact,enriquez2020covid,farrell2020consumption,findling2021serious,gerardi2021racial,gignac2021impacts,goda2022impact,goda2021impact,grantz2020use,haughwout2020us,horvath2021covid,li2020impact,loibl2020role,lusardi20186,moen2020disparate,mullen2022economic,quinby2021older,schneider2020household,schur2021covid,stavins2021unprepared}

\section{ Prior Literature}

 

\subsection{COVID-19 and  Financial Security  }

There is a burgeoning body of literature on the economic consequences of the COVID-19 pandemic for U.S. households. Most directly, several studies analyze changes in labor force participation and unemployment in response to the COVID-19 pandemic (e.g.  \cite{cheng2020back}), with a few studies focusing on labor trends  \citep{goda2021impact,moen2020disparate,quinby2021older}.   

Reductions in income may be offset in part by reductions in consumption  \citep{baker2020does,casado2020aggregate,chetty2020did,farrell2020consumption,horvath2021covid}. For example, \cite{farrell2020consumption} find an overall 10 percent decline in consumer spending following the onset of the pandemic. However, the effects on consumption are heterogeneous across a number of dimensions.  \cite{chetty2020did} find that much of the reduction in spending is concentrated among higher income households—households in the top income quartile spent 13 percent less as of mid-July 2020 relative to January 2020, whereas households in the bottom income quartile reduced consumption by only 4 percent during the same period. For those experiencing a COVID related loss of income, \cite{farrell2020consumption} find that receipt of pandemic-related unemployment benefits is associated with a 10 percent increase in consumer spending relative to the prior year. 


 
 
 Emerson and colleagues studied the effects of pandemic  on people with disabilities in the UK ‘Understanding Society’, an annual household panel study \citep{emerson2021impact}. Respondents with disability were more likely to experience higher levels of financial stress.  

 Gignac and colleagues studied people with physical or mental health disabilities in Canada. People with a  disability reported more  financial concerns, more contract work, and unmet accommodations at work  than those with no disability. \citep{gignac2021impacts}

  Goda, et al sue Current Population Survey and monthly Google Trends data to show fewer labor force exits due to disability and applications for disability insurance \citep{goda2022impact}


\citeauthor{schur2021covid} find White and Black women with disabilities experienced relatively greater employment losses during the pandemic compared to White men without disabilities. \citep{schur2021covid}

   \cite{cowan2020short} also finds workers with a disability—have experienced the largest declines in the likelihood of (full-time) work and work hours.

  
Kathleen Mullen and Nicole Maestas
 
Previous economic downturns have led to increases in applications for and, eventually, receipt of Social Security Disability Insurance (SSDI) benefits. In the pandemic-induced recession of 2020 and its aftermath, however, SSDI applications did not increase. One important factor may have been the prolonged closure of SSA field offices, since previous research finds that field office closures lead to persistent declines in SSDI beneficiaries in the surrounding communities. In this case, there may be pent-up demand for SSDI benefits as normal operations resume in areas where the economy has not fully recovered. Government support programs were materially different than in past recessions, which appears to have enabled people to weather the downturn and avoid applying for SSDI benefits. \cite{mullen2022economic}

 


 
 
\section{Data and Methods}
We explore the differences in financial wellbeing of households with and without disabilities in two different time periods. First, we use two datasets that are repeated cross sections, allowing us to compare households with and without disabilities before and after the start of the pandemic. Second, we use individual-level data that is recorded monthly starting in the spring of 2020. This tells us more about the evolution of the gap between households with and without disabilities throughout the pandemic. We explain each data and empirical strategy next. 

\subsection{Before and After Pandemic Changes}
We use two datasets to explore the differences in financial security for those with and without disabilities before and after the pandemic. 

First, we use the National Financial Capability Study (NFCS) data, a nationally-representative survey conducted every three years by the FINRA Investor Education Foundation. We use data from the 2018 and 2021 waves, as they include information on disability status, as well as financial wellbeing. The 2021 survey was conducted in the late summer to early fall. We consider a respondent to have a disability if they answered ``Permanently sick, disabled, or unable to work" to the question ``Which of the following best describes your current employment or work status?" We recognize that this is a relatively broad definition of disability. In order to examine a sample that is working-age, we limit our respondents to those between 18 and 64 years of age. 

We define five main variables of interest in this sample that all reflect the financial security of the household: presence of emergency savings, financial anxiety, whether they made a late credit card payment, whether they used alternative financial services, and the U.S. Consumer Financial Protection Bureau's financial wellbeing scale (FWB). The first four are dummy variables. FWB is measured from 0 to 100, based on the answers to five questions related to one's ability to keep up with day-to-day or month-to-month finances, as well as individual expectations of their ability to meet future financial goals.\footnote{For more on the FWB measure, see \url{https://www.consumerfinance.gov/consumer-tools/financial-well-being/about/}.}  


Second, we use data from the Survey of Household Economic Decisionmaking (SHED) provided by the Federal Reserve Board. MORE HERE. 



\subsection{Within Pandemic Changes}
Next, we turn to data from the Census Pulse....import all of Viv's stuff. 


 

\section{Findings}
\subsection{Before and After Pandemic Changes}
Beginning with the NFCS, Figure \ref{NFCS_Y} shows the difference in financial outcomes across 2018 and 2021 for respondents with and without disabilities. The changes are measured in percentage point differences, since each outcome is a dummy variable. While both groups were more likely to have emergency savings in 2021 than 2018, those without disabilities experienced a greater increase.\footnote{The full question is: Have you set aside emergency or rainy day funds that would cover your expenses for 3 months, in case of sickness, job loss, economic downturn, or other emergencies?} At the same time, financial anxiety increased by more for people without disabilities than people with disabilities. A similar story exists for AFS use, which includes payday loans, pawn shops, tax return advances, and rent-to-own services. However, the AFS measure asks about the last five years, making it potentially less of a pandemic story and a more overall trend. Both groups saw an equal increase in the likelihood of having a credit card with a late payment, conditional on having a credit card. Taken together, these findings suggest that though households are more likely to have three months of savings in case of emergencies, they are more  anxious about the future of their financial situation than they were three years prior. While households with disabilities were more likely to be financially anxious than before the pandemic started, the change in their experiences are less severe than people without disabilities. This could be because SSA's programs provide a guarantee of future income for households with disabilities. 

We next look at more subjective measures in Figure \ref{NFCS_FWB} in the CFPB's FWB score. Overall, people with disabilities saw an improvement in FWB, while people without disabilities saw a small decrease in FWB. The only measure within the scale that respondents with disabilities did worse on in 2021 than 2018 was thinking the money they had or will save won't last. This matches the financial anxiety increase, suggesting that there is concern for the future. However, people with disabilities were less likely to say they were just getting by financially, they will never have the things they want in life because of money, and their finances controlled their lives. They were more likely to say they had money left over at the end of the month, consistent with the finding from Figure \ref{NFCS_Y} that respondents with disabilities had more emergency savings in 2021 than 2018. 

Overall, these findings suggest that while the ability to keep up with month-to-month and day-to-day finances improved for respondents with disabilities from 2018 to 2021---particularly when compared to those without disabilities---they remain more anxious about their financial future than in the past. The 2021 survey took place over a year into the pandemic, but worldwide changes due to the pandemic could have made people more uncertain about the future. On every marker except for emergency savings, respondents with disabilities had less determintal changes from 2018 to 2021. 


\subsection{Within Pandemic Changes}

 
  

\section{Conclusion}
 
Implications for Social Security Programs
 
Increased eligibility for SSI and Medicaid, as well as SNAP and LIHEAP
 

\section{Figures}
 
 

\begin{figure}[h!]\label{NFCS_Y}
\caption{Changes in Financial Outcomes from 2018-2021 across People with and without Disabilities (NFCS)}
\centering
\includegraphics[scale=0.4]{Exhibits/ChangeY_18_21_NFCSdisabilitynodis.png}
\medskip 
\begin{minipage}{0.65\textwidth} 
{\footnotesize Notes: Data come from the 2018 and 2021 NFCS. Each bar represents the difference in the average measure for people with and without disabilities from 2018 to 2021 in percentage point terms. Emergency is whether someone has emergency savings. Financial anxiety is whether the individual agreed with the following statement: ``discussing my finances can make my heart race or make me feel stressed." Late CC Payment is whether someone was charged a late fee on their credit card. AFS use is whether the individual used a payday lender, pawn shop, tax return advance, or rent-to-own service in the past five years.   \par}
\end{minipage}
\end{figure}
  
\begin{figure}[h!]\label{NFCS_FWB}
\caption{Changes in Financial Wellbeing from 2018-2021 across People with and without Disabilities (NFCS) }
\centering
\includegraphics[scale=0.4]{Exhibits/ChangeFWB_18_21_NFCSdisabilitynodis.png}
\medskip 
\begin{minipage}{0.65\textwidth} 
{\footnotesize Notes: Data come from the 2018 and 2021 NFCS. Each bar represents the difference in the average measure for people with and without disabilities from 2018 to 2021. FWB measured from 0 to 100. Each component of FWB are the following five questions. These are scaled from 0 to 4, where higher scores are recoded to always be better. Q1: I am just getting by financially; Q2: I am concerned the money I have or will save won't last; Q3: Because of my money situation, I feel like I will never have the things I want in life; Q4: My finances control my life; Q5: I have money left over at the end of the month. \par}
\end{minipage}
\end{figure}
  


 
 


 
 \clearpage
\section{Tables}
 


 



 

\begin{table}[!h]
\caption{Stuff}
\label{tab:newtable}
 \scalebox{0.55}{   } }
\end{table} 
 
  



 
\clearpage


\section*{References}
\renewcommand*{\refname}{\vspace*{-12mm}} 

%%  bib file with citations saved as "refs.bib"
\bibliography{refs}
\bibliographystyle{chicago}

\section*{Appendix}



\clearpage
 
 
 
 
 
\includepdf[scale=1]{back.pdf}
 

\end{document}